\chapter{Statements}

% =======================================================================
% AssertStatement
% =======================================================================

\section{Assert Statement}

\begin{syntax}
  \verb+AssertStmt+ & $::=$ & \token{assert}\ \verb+Expr+\\
\end{syntax}

\paragraph{Description.}  Represents an {\em assert statement} of the form
``\lstinline{assert e}'', where \lstinline{e} is a \gls{boolean_expression}. 

\paragraph{Examples.} The following illustrates:
\begin{lstlisting}
function abs(int x) => int:
    if x < 0:
        x = -x
    assert x >= 0
    return x
\end{lstlisting}

\paragraph{Notes.}  Assertions are either {\em statically checked} by the
verifier, or turned into {\em runtime checks}.


% =======================================================================
% Assignment Statement
% =======================================================================

\section{Assignment Statement}

\begin{syntax}
  \verb+AssignStmt+ & $::=$ & \verb+LVal+\ \token{=}\ \verb+Expr+\\
\end{syntax}

\paragraph{Description.}  Represents an {\em assignment statement} of the
form \lstinline{lhs = rhs}.  Here, the \lstinline{rhs} is any
expression, whilst the \lstinline{lhs} must be an \lstinline{LVal} ---
that is, an expression permitted on the left-side of an assignment.

\paragraph{Examples.} The following illustrates different possible assignment statements:
\begin{lstlisting}
x = y       // variable assignment
x.f = y     // field assignment
x[i] = y    // list assignment
x[i].f = y  // compound assignment
\end{lstlisting}

The last assignment here illustrates that the left-hand side of an
assignment can be arbitrarily complex, involving nested assignments
into lists and records.

\paragraph{Semantics.}

\paragraph{Notes.} 

% =======================================================================
% Assume Statement
% =======================================================================

\section{Assume Statement}

\begin{syntax}
  \verb+AssumeStmt+ & $::=$ & \token{assume}\ \verb+Expr+\\
\end{syntax}

\paragraph{Description.}  Represents an {\em assume statement} of the form
``\lstinline{assume e}'', where \lstinline{e} is a \gls{boolean_expression}. 

\paragraph{Examples.}  The following illustrates a simple function
which uses an \lstinline{assume} statement to meet its
postcondition:

\begin{lstlisting}
function abs(int x) => int:
    assume x >= 0
    return x
\end{lstlisting}

\paragraph{Notes.} Assumptions are {\em assumed} by the verifier and, since
this may be unsound, are always turned into {\em runtime checks}.

% =======================================================================
% ReturnStatement
% =======================================================================

\section{Return Statement}

\begin{syntax}
  \verb+ReturnStmt+ & $::=$ & \token{return}\ \big[\ \verb+Expr+\
  \big]\\
\end{syntax}

\paragraph{Description.}  Represents a {\em return statement} with an
optional expression is referred to as the {\em return value}.

\paragraph{Examples.} The following illustrates a simple function
which returns the increment of its parameter \lstinline{x}:

\begin{lstlisting}
function f(int x) => int:
    return x + 1
\end{lstlisting}

Here, we see a simple \lstinline{return} statement which returns an \lstinline{int} value.

\paragraph{Notes.}  The returned expression (if there is one) must
begin on the same line as the return statement itself.

% =======================================================================
% ThrowStatement
% =======================================================================

\section{Throw Statement}

\begin{syntax}
  \verb+ThrowStmt+ & $::=$ & \token{throw}\ \verb+Expr+\\
\end{syntax}

\paragraph{Description.}

\paragraph{Examples.}

\paragraph{Notes.} 

% =======================================================================
% Variable Declarations
% =======================================================================

\section{Variable Declarations}

\begin{syntax}
  \verb+VarDecl+ & $::=$ & \verb+Type+\ \verb+Ident+\ \big[\
  \token{=}\ \verb+Expr+\ \big]\\
\end{syntax}

\paragraph{Description.}  Represents a \gls{variable_declaration}
which has an optional expression assignment referred to as an
\gls{variable_initialiser}.  If an initialiser is given, then this will be evaluated and assigned to the variable when the declaration is executed.

\paragraph{Examples.} Some example variable declarations are:
\begin{lstlisting}
int x
int y = 1
int z = x + y
\end{lstlisting}

\paragraph{Notes.} 


% =======================================================================
% If Statement
% =======================================================================

\section{If Statement}

\begin{syntax}
  \verb+IfStmt+$^\ell$ & $::=$ & \token{if}\ \verb+Expr+\ \token{:}
  \verb+Block+$^\gamma$\ \big[ \token{else} \token{:}
  \verb+Block+$^\omega$\ \big]\\
&&\\
\multicolumn{3}{r}{(where $\ell < \gamma$ and $\ell < \omega$)}\\
\end{syntax}

\paragraph{Description.}

\paragraph{Examples.}

\paragraph{Notes.} 

% =======================================================================
% While Statement
% =======================================================================

\section{While Statement}

\begin{syntax}
  \verb+WhileStmt+$^\ell$ & $::=$ & \token{while}\ \verb+Expr+\ \token{:}
  \verb+Block+$^\gamma$\ \big(\ \token{where}\ \verb+Expr+\ \big)$^*$\\
&&\\
\multicolumn{3}{r}{(where $\ell < \gamma$)}\\
\end{syntax}

\paragraph{Description.}

\paragraph{Examples.}

\paragraph{Notes.} 

% =======================================================================
% Do/While Statement
% =======================================================================

\section{Do/While Statement}

\begin{syntax}
  \verb+DoWhileStmt+$^\ell$ & $::=$ & \token{do}\ \token{:}
  \verb+Block+$^\gamma$\ \token{while}\ \verb+Expr+\ \big(\
  \token{where}\ \verb+Expr+\ \big)$^*$\\
&&\\
\multicolumn{3}{r}{(where $\ell < \gamma$)}\\
\end{syntax}

\paragraph{Description.}

\paragraph{Examples.}

\paragraph{Notes.} 

% =======================================================================
% For Statement
% =======================================================================

\section{For Statement}

\begin{syntax}
  \verb+ForStmt+ & $::=$ &\\
\end{syntax}

\paragraph{Description.}

\paragraph{Examples.}

\paragraph{Notes.} 

% =======================================================================
% Switch Statement
% =======================================================================

\section{Switch Statement}

\begin{syntax}
  \verb+SwitchStmt+ & $::=$ &\\
\end{syntax}

\paragraph{Description.}

\paragraph{Examples.}

\paragraph{Notes.} 

% =======================================================================
% Try/Catch Block
% =======================================================================

\section{Try/Catch Statement}

\begin{syntax}
  \verb+TryCatchStmt+ & $::=$ &\\
\end{syntax}

\paragraph{Description.}

\paragraph{Examples.}

\paragraph{Notes.} 
