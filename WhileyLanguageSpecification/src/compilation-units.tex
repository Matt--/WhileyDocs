\chapter{Source Files}

Whiley programs are split across one or more \gls{source_file}s which
are compiled into \gls{wyil_file}s prior to execution.
\Gls{source_file}s contain declarations which describe the functions,
methods, data types and constants which form the program.
\Gls{source_file}s are grouped together into coherent units called
\gls{package}s.


\section{Compilation Units}

\section{Packages \& Imports}

\section{Declarations}

Camel case

\subsection{Access Control}

% =======================================================================
% Type Declarations
% =======================================================================

\subsection{Type Declarations}

A {\em type declaration} declares a named type within a Whiley
\gls{source_file}.  The declaration may refer to named types in this
or other source filess and may also {\em recursively}
refer to itself (either directly or indirectly).

\begin{syntax}
  \verb+TypeDecl+ & $::=$ & \token{type}\ \verb+Ident+\ \token{is}\
  \verb+TypePattern+\ \big[\ \token{where}\ \verb+Expr+\ \big]\\
\end{syntax}

The optional \lstinline{where} clause defines a
\gls{boolean_expression} which holds for any instance of this type.
This is often referred to as the type's {\em constraint} or {\em
  invariant}.  Variables declared within the {\em type pattern} may be
referred to within the optional \lstinline{where} clause.

\paragraph{Examples.}  Some simple examples illustrating type
declarations are:

\begin{lstlisting}
// Define a simple point type
type Point is { int x, int y }

// Define the type of natural numbers
type nat is (int x) where x >= 0
\end{lstlisting}

The first declaration defines an unconstrained record type named
\lstinline{Point}, whilst the second defines a constrained integer
type \lstinline{nat}.

\paragraph{Notes.}  A convention is that type declarations for {\em
  records} or {\em unions of records} begin with an upper case
character (e.g. \lstinline{Point} above).  All other type declarations
begin with lower case.  This reflects the fact that records are most
commonly used to describe objects in the domain.

% =======================================================================
% Constant Declarations
% =======================================================================

\subsection{Constant Declarations}

A {\em constant declaration} declares a named constant within a Whiley
\gls{source_file}.  The declaration may refer to named constants in this
or other source filess, although it may not refer to itself (either
directly or indirectly).

\begin{syntax}
  \verb+ConstantDecl+ & $::=$ & \token{constant}\ \verb+Ident+\
  \token{is}\ \verb+Expr+\\
\end{syntax}

The given {\em constant expression} is evaluated at {\em compile time}
and must produce a constant value.  This prohibits the use of function
or method calls within the constant expression.  However, general
operators (e.g. for arithmetic) are permitted.

\paragraph{Examples.}  Some example to illustrate constant
declarations are:

\begin{lstlisting}
// Define the well-known mathematical constant to 10 decimal places.
constant PI is 3.141592654

// Define a constant expression which is twice PI
constant TWO_PI is PI * 2.0
\end{lstlisting}

The first declaration defines the constant \lstinline{PI} to have the
\lstinline{real} value \lstinline{3.141592654}.  The second
declaration illustrates a more interesting constant expression which
is evaluated to \lstinline{6.283185308} at compile time.

\paragraph{Notes.}  A convention is that constants are named in upper
case with underscores separating words (i.e. as in \lstinline{TWO_PI}
above).

% =======================================================================
% Function Declarations
% =======================================================================

\subsection{Function Declarations}

\begin{syntax}
  \verb+FunctionDecl+ & $::=$ & \token{function}\ \verb+Ident+\
  \verb+TypePattern+\ \token{=>}\ \verb+TypePattern+\ \big(\\
  && \ \ \token{throws}\ \verb+Type+\ $|$\ \token{requires}\
  \verb+Expr+\ $|$\ \token{ensures}\ \verb+Expr+\\
  && \big)$^*$\ \token{:}\ \verb+Block+\\
\end{syntax}

\paragraph{Description.}

\paragraph{Examples.}

\paragraph{Notes.}

% =======================================================================
% Method Declarations
% =======================================================================

\subsection{Method Declarations}

\begin{syntax}
  \verb+MethodDecl+ & $::=$ & \token{method}\ \verb+Ident+\
  \verb+TypePattern+\ \token{=>}\ \verb+TypePattern+\ \big(\\
  && \ \ \token{throws}\ \verb+Type+\ $|$\ \token{requires}\
  \verb+Expr+\ $|$\ \token{ensures}\ \verb+Expr+\\
  && \big)$^*$\ \token{:}\ \verb+Block+\\
\end{syntax}

\paragraph{Description.}

\paragraph{Examples.}

\paragraph{Notes.}




