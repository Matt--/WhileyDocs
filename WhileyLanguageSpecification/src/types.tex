\chapter{Types}
\section{Overview}
Discuss syntactic versus semantic types.

\begin{syntax}
  \verb+Type+ & $::=$ & &\\
  & $|$ & \verb+TermType+ &\\
  & $|$ & \verb+UnionType+ &\\
  & $|$ & \verb+IntersectionType+ &\\
\end{syntax}

\begin{syntax}
  \verb+TermType+ & $::=$ & &\\
  & $|$ & \verb+PrimitiveType+ &\\
  & $|$ & \verb+TupleType+ &\\
  & $|$ & \verb+RecordType+ &\\
  & $|$ & \verb+ReferenceType+ &\\
  & $|$ & \verb+NominalType+ &\\
  & $|$ & \verb+CollectionType+ &\\
  & $|$ & \verb+NegationType+ &\\
\end{syntax}


\section{Primitives}

\begin{syntax}
  \verb+PrimitiveType+ & $::=$ & &\\
  & $|$ & \verb+AnyType+ &\\
  & $|$ & \verb+VoidType+ &\\
  & $|$ & \verb+NullType+ &\\
  & $|$ & \verb+BoolType+ &\\
  & $|$ & \verb+CharType+ &\\
  & $|$ & \verb+IntType+ &\\
  & $|$ & \verb+RealType+ &\\
\end{syntax}


% =======================================================================
% Any
% =======================================================================

\subsection{Any Type}

\begin{syntax}
  \verb+AnyType+ & $::=$ & \token{any} &\\
\end{syntax}

\paragraph{Description.}  The type \lstinline{any} represents the type whose variables may hold any possible value.

\paragraph{Examples.}

\paragraph{Semantics.}

\paragraph{Notes.} The \lstinline{any} type is top in the type
lattice.  That is, it is the supertype of all other types.

% =======================================================================
% Void 
% =======================================================================

\subsection{Void Type}

\begin{syntax}
   \verb+VoidType+ & $::=$ & \token{void} &\\
\end{syntax}

\paragraph{Description.} The \lstinline{void} type represents the type whose variables cannot exist! That is, they cannot hold any possible value. Void is used to represent the return type of a function which does not return anything. However, it is also
used to represent the element type of an empty list of set. 

\paragraph{Examples.}

\paragraph{Semantics.}

\paragraph{Notes.} The void type is a subtype of everything; that is, it is bottom in the type lattice.

% =======================================================================
% Null
% =======================================================================

\subsection{Null Type}

\begin{syntax}
  \verb+NullType+ & $::=$ & \token{null} &\\
\end{syntax}

\paragraph{Description.}  The null type is a special type which should
be used to show the absence of something. It is distinct from void,
since variables can hold the special \lstinline{null}> value (where as
there is no special "\lstinline{void}" value).

\paragraph{Examples.}

\paragraph{Semantics.}

\paragraph{Notes.}  With all of the problems surrounding
\lstinline{null} and \lstinline{NullPointerException}s in languages
like Java and C, it may seem that this type should be
avoided. However, it remains a very useful abstraction to have around
and, in Whiley, it is treated in a completely safe manner (unlike
e.g. Java).

% =======================================================================
% Bool 
% =======================================================================

\subsection{Bool Type}

\begin{syntax}
 \verb+BoolType+ & $::=$ & \token{bool} &\\
\end{syntax}

\paragraph{Description.}

\paragraph{Examples.}

\paragraph{Semantics.}

\paragraph{Notes.} 

% =======================================================================
% Char
% =======================================================================

\subsection{Char Type}

\begin{syntax}
  \verb+CharType+ & $::=$ & \token{char} & \\
\end{syntax}

\paragraph{Description.}

\paragraph{Examples.}

\paragraph{Semantics.}

\paragraph{Notes.} 

% =======================================================================
% Int
% =======================================================================

\subsection{Int Type}

\begin{syntax}
  \verb+IntType+ & $::=$ & \token{int} &\\
\end{syntax}

\paragraph{Description.}

\paragraph{Examples.}

\paragraph{Semantics.}

\paragraph{Notes.} 

% =======================================================================
% Real
% =======================================================================

\subsection{Real Type}

\begin{syntax}
  \verb+RealType+ & $::=$ & \token{real} &\\
\end{syntax}

\paragraph{Description.}

\paragraph{Examples.}

\paragraph{Semantics.}

\paragraph{Notes.} 

% =======================================================================
% Tuples
% =======================================================================

\section{Tuple Types}

\begin{syntax}
  \verb+TupleType+ & $::=$ & \token{(}\ \verb+Type+\ \big(\ \token{,}\
  \verb+Type+\ \big)$^*$\ \token{)}&\\
\end{syntax}

\paragraph{Description.}

\paragraph{Examples.}

\paragraph{Semantics.}

\paragraph{Notes.}

% =======================================================================
% Records
% =======================================================================

\section{Record Types}

\begin{syntax}
  \verb+RecordType+ & $::=$ & \token{\{}\ \verb+Type+\
  \verb+Identifier+\ \big(\ \token{,}\ \verb+Type+\ \verb+Identifier+\ \big)$^*$ \token{\}}&\\
\end{syntax}

\paragraph{Description.}

\paragraph{Examples.}

\paragraph{Semantics.}

\paragraph{Notes.}

% =======================================================================
% References
% =======================================================================

\section{Reference Types}

\begin{syntax}
  \verb+ReferenceType+ & $::=$ & \token{\&}\ \ \verb+Type+&\\
\end{syntax}

\paragraph{Description.}

\paragraph{Examples.}

\paragraph{Semantics.}

\paragraph{Notes.}

% =======================================================================
% Nominal
% =======================================================================

\section{Nominal Types}

\begin{syntax}
  \verb+NominalType+ & $::=$ & \verb+Identifier+&\\
\end{syntax}

\paragraph{Description.}

\paragraph{Examples.}

\paragraph{Semantics.}

\paragraph{Notes.}

% =======================================================================
% Collections
% =======================================================================

\section{Collection Types}

% =======================================================================
% Set
% =======================================================================

\subsection{Set Type}

\begin{syntax}
  \verb+SetType+ & $::=$ & \token{\{} \ \verb+Type+ \ \token{\}} &\\
\end{syntax}

\paragraph{Description.}

\paragraph{Examples.}

\paragraph{Semantics.}

\paragraph{Notes.} 

% =======================================================================
% Map
% =======================================================================

\subsection{Map Type}

\begin{syntax}
  \verb+MapType+ & $::=$ & \token{\{} \ \verb+Type+ \ \token{=>} \ \verb+Type+ \ \token{\}} &\\
\end{syntax}

\paragraph{Description.}

\paragraph{Examples.}

\paragraph{Semantics.}

\paragraph{Notes.} 

% =======================================================================
% List
% =======================================================================

\subsection{List Type}

\begin{syntax}
  \verb+ListType+ & $::=$ & \token{[} \ \verb+Type+ \ \token{]}&\\
\end{syntax}

\paragraph{Description.}

\paragraph{Examples.}

\paragraph{Semantics.}

\paragraph{Notes.} 

% =======================================================================
% Unions
% =======================================================================

\section{Union Types}

\begin{syntax}
  \verb+UnionType+ & $::=$ & \verb+IntersectionType+\ \big(\ \token{|}\ \verb+IntersectionType+\
  \big)$^+$&\\
\end{syntax}

\paragraph{Description.}

\paragraph{Examples.}

\paragraph{Semantics.}

\paragraph{Notes.}

% =======================================================================
% Intersections
% =======================================================================

\section{Intersection Types}

\begin{syntax}
  \verb+IntersectionType+ & $::=$ & \verb+TermType+\ \big(\ \token{\&}\ \verb+TermType+\
  \big)$^+$&\\
\end{syntax}

\paragraph{Description.}

\paragraph{Examples.}

\paragraph{Semantics.}

\paragraph{Notes.}

% =======================================================================
% Negations
% =======================================================================

\section{Negation Types}

\begin{syntax}
  NegationType & $::=$ & \token{!}\ \ \verb+Type+&\\
\end{syntax}

\paragraph{Description.}

\paragraph{Examples.}

\paragraph{Semantics.}

\paragraph{Notes.}

\section{Subtyping}
Discussion or present subtyping algorithm?
