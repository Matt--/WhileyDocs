\chapter{Introduction}

This document provides a specification of the {\em Whiley Programming
  Language}.  Whiley is a hybrid imperative and functional programming
language designed to produce programs with fewer errors that those
developed by more convention means.  Whiley allows explicit
specifications to be given for functions, methods and data structures,
and employs a \gls{verifying_compiler} to check whether programs meet their specifications.  As such, Whiley is ideally suited
for use in \gls{safety_critical_system}s.  However, there are many
benefits to be gained from using Whiley in a general setting
(e.g. improved documentation, maintainability, reliability, etc).
Finally, this document is {\em not} intended as a general introduction
to the language, and the reader is referred to alternative documents
for learning the language~\cite{X}.


\section{Background}

Reliability of large software systems is a difficult problem facing
software engineering, where subtle errors can have disastrous
consequences.  Infamous examples include: the Therac-25 disaster where
a computer-operated X-ray machine gave lethal doses to
patients~\cite{LT93}; the 1988 worm which reeked havoc on the internet
by exploiting a buffer overrun~\cite{ER89}; the 1991 Patriot missile
failure where a rounding error resulted in the missile catastrophically
hitting a barracks~\cite{GAO}; and, the Ariane 5 rocket which exploded
shortly after launch because of an integer overflow, costing the ESA
an estimated \$500 million~\cite{ARIAN5}.

% The most widely used and accepted approach to improving software
% reliability is through extensive testing and manual code inspection.
% Whilst this does increase confidence, it cannot guarantee the absence
% of errors --- which is particularly problematic in a safety-critical
% setting.  Another successful approach is to prove the correctness of
% {\em models of software}, rather than of the software itself.  For
% example, model checkers
% (e.g~\cite{Clarke99,Holz01,BR02}) and SAT solvers % Mcm93,,CDHLPZ00
% (e.g.~\cite{MMZZM01,MFM04}) have proved highly % ,HDZM04,LS04,VB03
% effective at checking correctness properties of finite models of
% software systems, including microprocessor
% designs~\cite{VB03,Schub03}, flight-control
% systems~\cite{GH02,Choi05}, network protocols~\cite{Boli98,AC03} and
% spaceflight-control systems~\cite{HP00b}.  Some model checkers % ,NP02
% (e.g. CBMC~\cite{CKL04}, Java Pathfinder~\cite{HP00b},
% BLAST~\cite{HJMS03}, SLAM~\cite{BMMR01}) can also be applied directly
% on the program code although, in such cases, either significant
% abstraction is performed (hence, reducing the scope) or scalability is
% sacrificed.

Around 2003, Hoare proposed the creation of a {\em verifying compiler} as a grand challenge for computer science \cite{Hoare03}.  A verifying
compiler ``{\em uses automated mathematical and logical reasoning to
  check the correctness of the programs that it compiles.}''  There
have been numerous attempts to construct a verifying compiler system,
although none has yet made it into the mainstream.  Early examples
include that of King~\cite{King69}, Deutsch~\cite{Deutsch73}, the
Gypsy Verification Environment~\cite{Good85} and the Stanford Pascal
Verifier~\cite{LGHKMOPS95}.  More recently, the Extended Static
Checker for Modula-3~\cite{DLNS98} which became the Extended Static
Checker for Java (ESC/Java) --- a widely acclaimed and influential
work~\cite{FLLNSS02}.  Building on this success was JML and its
associated tooling which provided a standard notation for specifying
functions in Java~\cite{LCCRC05}.  Finally, Microsoft %,CR08,JC10
developed the Spec\# system which is built on top of
C\#~\cite{BLS04}.% ,BDFLS04,BCDJL06

\section{Goals}

The Whiley Programming Language has been designed from scratch in
conjunction with a verifying compiler.  The intention of this is to
provide an open framework for research in automated software
verification.  The initial goal is to automatically eliminate common
errors, such as {\em null dereferences}, {\em array-out-of-bounds},
{\em divide-by-zero} and more.  In the future, the intention is to
consider more complex issues, such as termination, proof-carrying code and user-supplied proofs.

\section{History}