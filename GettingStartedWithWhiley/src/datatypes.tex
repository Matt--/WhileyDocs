\newpage
\section{Basic Concepts}

\subsection{Primitives}

\subsection{Sets, Lists and Maps}

\subsection{Records and Tuples}

\subsection{Primitive Subtyping.} 
Whiley also provides strong guarantees regarding subtyping of
primitive types (i.e. integers and reals).  In Whiley,
\lstinline{int}s and \lstinline{real}s represent unbounded integers
and rationals, which ensures \lstinline{int}$\;\le\;$\lstinline{real}
has true subset semantics (i.e. every \lstinline{int} can be
represented by a \lstinline{real}).  This is not true for e.g. Java,
where there are \lstinline{int} (resp. \lstinline{long}) values which
cannot be represented using \lstinline{float}
(resp. \lstinline{double})~\cite[\S5.1.2]{GJSB05}.

\subsection{Value Semantics}
\label{value_semantics}
In Whiley, all compound structures (e.g. lists, sets, and records)
have {\em value semantics}.  This means they are passed and returned
by-value (as in Pascal, MATLAB or most functional languages).  But, unlike
functional languages (and like Pascal), values of compound types can
be updated in place.

Value semantics implies that updates to a variable only affects that
variable, and that information can only flow out of a function through
its return value.  Whiley has no general, mutable heap comparable to
those found in object-oriented languages.  Consider:
\begin{lstlisting}
 int f([int] xs):
     ys = xs
     xs[0] = 1
     ...
\end{lstlisting}
The semantics of Whiley dictate that, having assigned \lstinline{xs}
to \lstinline{ys} as above, the subsequent update to \lstinline{xs}
does not affect \lstinline{ys}.  Arguments are also passed by value,
hence \lstinline{xs} is updated inside \lstinline{f()} and this does
not affect \lstinline{f}'s caller.  That is, \lstinline{xs} is not a
{\em reference} to a list of \lstinline{int}; rather, it {\em is} a
list of \lstinline{int}s and assignments to it do not affect state
visible outside of \lstinline{f()}. 

Whilst this approach may seem inefficient, a variety of techniques
exist (e.g. reference counting) to ensure efficiency (see
e.g.~\cite{LH11,Shank01,Ode91}).  Indeed, the underlying
implementation does pass compound structures by reference and copies
them only when absolutely necessary.

