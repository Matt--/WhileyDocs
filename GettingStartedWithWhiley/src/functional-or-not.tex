\newpage
\section{Functions, Methods and Objects}



\subsection{Function Purity}
Much research has been done on functional purity for object-oriented
languages (e.g.~\cite{Pea11,Rou04,SR05,MRR02}), because it makes 
many things more tractable, including: {\em automatic
 parellelisation}~\cite{ABCR10,CRPAHBW10}, {\em software
 verification}~\cite{Leav02,BNSS04,BA05,DL07}, {\em query
 systems}~\cite{LHS97,WPN08}, {\em compiler
 optimisations}~\cite{Cla97,LLH05,ZRKW08} and more.

\subsection{Value Semantics}
\label{value_semantics}
In Whiley, all compound structures (e.g. lists, sets, and records)
have {\em value semantics}.  This means they are passed and returned
by-value (as in Pascal, MATLAB or most functional languages).  But, unlike
functional languages (and like Pascal), values of compound types can
be updated in place.

Value semantics implies that updates to a variable only affects that
variable, and that information can only flow out of a function through
its return value.  Whiley has no general, mutable heap comparable to
those found in object-oriented languages.  Consider:
\begin{lstlisting}
 int f([int] xs):
     ys = xs
     xs[0] = 1
     ...
\end{lstlisting}
The semantics of Whiley dictate that, having assigned \lstinline{xs}
to \lstinline{ys} as above, the subsequent update to \lstinline{xs}
does not affect \lstinline{ys}.  Arguments are also passed by value,
hence \lstinline{xs} is updated inside \lstinline{f()} and this does
not affect \lstinline{f}'s caller.  That is, \lstinline{xs} is not a
{\em reference} to a list of \lstinline{int}; rather, it {\em is} a
list of \lstinline{int}s and assignments to it do not affect state
visible outside of \lstinline{f()}. 

Whilst this approach may seem inefficient, a variety of techniques
exist (e.g. reference counting) to ensure efficiency (see
e.g.~\cite{LH11,Shank01,Ode91}).  Indeed, the underlying
implementation does pass compound structures by reference and copies
them only when absolutely necessary.

\subsection{Objects and References}


%\subsection{Polymorphism \& Encapsulation}
%Two important hallmarks of the object-oriented paradigm are
%polymorphism and encapsulation.  The former is typically achieved
%through inheritance and/or interfaces, whilst the latter typically
%exploits \lstinline{public} / \lstinline{private} modifiers.  Whiley
%does not permit \lstinline{public} or \lstinline{private} modifiers in
%records.  Instead, Whiley employs a simple form of {\em existential
%type}~\cite{}, illustrated as follows:
%
%\begin{lstlisting}
% define Shape as { 
%   bool contains(int x, int y, Shape this), 
%   ... 
% }
%\end{lstlisting}
%This declares what is, essentially, an interface.  The
%``\lstinline{...}''  notation is significant here, as it denotes
%unknown --- or {\em existential} --- state~\cite{Cook09}.  We can then
%``implement'' this interface as follows:
%
%\begin{lstlisting}
% define Square as {
%   int x, int y, int width, int height,   
%   bool contains(int x, int y, Square this)
% }
%
% bool defSqContains(int x, int y, Square sq):
%     if x >= sq.x && x < (sq.x + sq.width):
%         return y >= sq.y && 
%                y < (sq.y + sq.height)
%     return false
%
% Square defSquare(int x, int y, int w, int h):
%     return {x: x, y: y, width: w, height: h,
%             contains: &defSqContains}
%\end{lstlisting}
%{\bf THIS NEEDS WORK --- IT'S BROKEN}

\subsection{Simulating Interfaces}
\subsection{Concurrency}