\section{Calculator Example}
\begin{figure}[!p]
\begin{lstlisting}[numbers=left]
 define Var as string
 define Op as { ADD, SUB, MUL, DIV }
 define BinOp as { Op op, Expr lhs, Expr rhs } 
 define ListAccess as { Expr lhs, Expr rhs } 

 define Value as int | [Value] | null

 define Expr as int | Var | BinOp | [Expr] | ListAccess 

 Value eval(Expr e, {Var->Value} env):
    if e is int:
        return e
    else if e is Var && e in env:
        // look up variable's value
        return env[e]
    else if e is BinOp:
        // evaluate left and right expressions
        lhs = eval(e.lhs,env)
        rhs = eval(e.rhs,env)
        // sanity check
        if !(lhs is int && rhs is int):
            return null // stuck
        // evaluate result
        switch e.op:
            case ADD:
                return lhs + rhs
            case SUB:
                return lhs - rhs
            case MUL:
                return lhs * rhs
            case DIV:
                if rhs != 0:
                    return lhs / rhs
    else if e is ListAccess:
        // evaluate src and index expressions
        src = eval(e.lhs,env)
        index = eval(e.rhs,env)
        // santity check
        if src is [Value] && index is int
           && index >= 0 && index < |src|:
            return src[index]
    else if e is [Expr]:
        lv = []
        // evaluate items in list constructor
        for i in e:
            v = eval(i,env)
            if v == null:
                return v
            else:
                lv = lv + [v]
        return lv
    // some kind of error occurred, so propagate upwards
    return null
\end{lstlisting}
\caption{Whiley code for a simple expression tree and evaluation
  function.  This makes extensive use of type tests, both for
  distinguishing expressions and error handling.  Flow-sensitive
  typing greatly simplifies the code, which would otherwise require
  numerous unnecessary casts.}
\label{eg1}
\end{figure}

Figure~\ref{eg1} provides a simple implementation of expressions,
along with code for evaluating them.  The types \lstinline{Expr} and
\lstinline{Value} are algebraic data types, with the latter defining
the set of allowed values.  Type \lstinline{Op} is an enumeration,
whilst \lstinline{BinOp} and \lstinline{ListAccess} are records which
form part of \lstinline{Expr}.  Parameter \lstinline{env} is a map
from variables to \lstinline{Values}.  Finally, \lstinline{null} is
used as an error condition to indicate a ``stuck'' state (i.e. the
evaluation cannot proceed).

The code in Figure~\ref{eg1} makes extensive use of runtime type tests
to distinguish different expression forms (e.g.  
``\lstinline{e is int}'').  These work in a similar fashion to Java's
\lstinline{instanceof} operator, with one important difference: they
operate in a flow-sensitive fashion and automatically {\em retype}
variables after the test.  As an example, consider the type test
``\lstinline{e is int}'' on Line 11.  On the true branch, variable
\lstinline{e} is automatically retyped to have type \lstinline{int}.
Likewise, on the false branch, \lstinline{e} is now known {\em not} to
have type \lstinline{int} (and any attempt to retest this yields a
compile-time error).

Figure~\ref{eg1} also employs runtime type tests to identify and
propagate errors.  For example, having evaluated the left- and
right-hand sides of a \lstinline{BinOp}, we check on Line 21 that both
are \lstinline{int} values (i.e. not list values or
\lstinline{null}).  After the check, Whiley's flow-sensitive type
system automatically retypes both \lstinline{lhs} and \lstinline{rhs}
to \lstinline{int}.  For \lstinline{ListAccess} expressions, we check
on Line~39 that \lstinline{src} is a list value, and that
\lstinline{index} is an \lstinline{int}.  The latter is achieved with
``\lstinline{index is int}''.  As \lstinline{src} is retyped within the
condition itself, the subsequent use of \lstinline{|src|} on Line~40
is type safe.

Implementing our expression language in a statically-typed language,
such as Java, would require code that was more cumbersome, and more verbose
than that of Figure~\ref{eg1}.  One reason for this is that, in
languages like Java, variables must be {\em explicitly} retyped after
\lstinline{instanceof} tests.  That is, we must insert casts to update
the types of tested variables and, since variables can have only one
type in Java, introduce temporary variables to hold these new types.
For example, after a test ``\lstinline{e instanceof BinOp}'' we must
introduce a new variable, say \lstinline{r}, with type
\lstinline{BinOp} and assign \lstinline{e} to \lstinline{r} using an
appropriate cast.  A Java implementation would also (most likely)
break up the test on Line 39, since it would otherwise need two
identical casts (one inside the condition for \lstinline{|src|}, and
one on the true branch for \lstinline{src[index]}).

In an object-oriented language, such as Java, a direct conversion of
Figure~\ref{eg1} might not be optimal.  Instead, the {\em visitor
  pattern}~\cite{gofbook} can be used to distinguish different
expression forms.  Using the visitor pattern reduces the amount of
explicit retyping required.  This is because the different expression
forms are explicitly given as parameters to the visitor methods.
However, using the visitor pattern is a heavyweight solution which is
not suitable in all situations.  In particular, it would not eliminate
all forms of explicit retyping from Figure~\ref{eg1}.  In this case,
explicit variable retyping will still be required to properly handle
the different values returned from \lstinline{eval()}.  For example,
to check \lstinline{BinOp}s are evaluated on \lstinline{int} operands
(Line~21), and that \lstinline{src} gives a list and \lstinline{index}
an \lstinline{int} (Line~39).

%% Functional programming languages provide {\em pattern matching} to
%% distinguish different forms of algebraic data type~\cite{duny}.  Like
%% the visitor pattern, this can be used to reduce the amount of explicit
%% retyping that is required.  {\bf [WHAT ELSE TO SAY HERE?]}


\paragraph{Code Reuse.}
Whiley's flow type system can expose greater opportunities
for code reuse:
\begin{lstlisting}[numbers=left]
{string} usedVariables(Expr e):
    if e is Var:
        return {e}
    else if e is BinOp || e is ListAccess:
        l = useVariables(e.lhs)
        r = useVariables(e.rhs)
        return l + r  // set union
    else if e is [Expr]:
        ...
    else:
        return {}
\end{lstlisting}

On Line 5, variable \lstinline{e} has type
\lstinline{BinOp|ListAcccess}.  The use of \lstinline{e.lhs} at this
point is type safe, since we can perform operations common to all
types of a union and, in particular, unions of records expose common
fields (similar to a {\em common initial sequence} for
\lstinline{union}s of \lstinline{struct}s in
C~\cite[\S6.3.2.3]{ISO90}).

In languages like Java, exploiting code reuse in this way requires
careful planning, as common types must be explicitly related in the
class hierarchy.  In contrast, Whiley's flow-sensitive type system
lets us exploit opportunities for code reuse in an ad-hoc fashion, as
and when they occur.

\subsubsection{Primitive Subtyping.} 
Whiley also provides strong guarantees regarding subtyping of
primitive types (i.e. integers and reals).  In Whiley,
\lstinline{int}s and \lstinline{real}s represent unbounded integers
and rationals, which ensures \lstinline{int}$\;\le\;$\lstinline{real}
has true subset semantics (i.e. every \lstinline{int} can be
represented by a \lstinline{real}).  This is not true for e.g. Java,
where there are \lstinline{int} (resp. \lstinline{long}) values which
cannot be represented using \lstinline{float}
(resp. \lstinline{double})~\cite[\S5.1.2]{GJSB05}.

