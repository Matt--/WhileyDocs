\section{Verification}

As discussed in the introduction, an important feature of Whiley is
{\em verification}.  That is made up of two aspects: firstly, the
ability to write specifications for functions and methods in Whiley;
secondly, the ability of the compiler to check the body of a function
or method meets its specification.

Unfortunately, specifications is not always straightforward and
requires considerable attention to detail.  Nevertheless, with
practice, it can easily fit into the routine of day-to-day
development.  In this section, we'll explore the basics of
verification in Whiley and, in the following section, we'll look at
practical example.

\paragraph{Example.}  To illustrate verification in Whiley, we'll
consider specifying a simple example function.  This is the
\lstinline{contains()} function, described as follows:

\begin{lstlisting}
// Returns true if the items list contains the given item and false otherwise.
function contains([int] items, int item) => bool:
    ...
\end{lstlisting}

This is a common function found in the standard libraries of many
programming languages.  The body of the function examines each element
of the \lstinline{items} list and check whether or not it equals
\lstinline{item}.  To start with, we won't worry too much about the
body of the \lstinline{contains()} function.  Instead, we'll
progressively build up the specification until we are happy with it.
Then, we'll give an implementation of the function which meets this
specification.


